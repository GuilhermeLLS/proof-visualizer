\documentclass[12pt, a4paper]{article}
\usepackage[utf8]{inputenc}
\usepackage{hyperref}

\usepackage{listings}
\usepackage{color}
\definecolor{lightgray}{rgb}{.9,.9,.9}
\definecolor{darkgray}{rgb}{.4,.4,.4}
\definecolor{purple}{rgb}{0.65, 0.12, 0.82}
\lstdefinelanguage{JavaScript}{
  keywords={typeof, new, true, false, catch, function, return, null, catch, switch, var, if, in, while, do, else, case, break},
  keywordstyle=\color{blue}\bfseries,
  ndkeywords={class, export, boolean, throw, implements, import, this},
  ndkeywordstyle=\color{darkgray}\bfseries,
  identifierstyle=\color{black},
  sensitive=false,
  comment=[l]{//},
  morecomment=[s]{/*}{*/},
  commentstyle=\color{purple}\ttfamily,
  stringstyle=\color{red}\ttfamily,
  morestring=[b]',
  morestring=[b]"
}
\lstset{
   language=JavaScript,
   backgroundcolor=\color{lightgray},
   extendedchars=true,
   basicstyle=\footnotesize\ttfamily,
   showstringspaces=false,
   showspaces=false,
   numbers=left,
   numberstyle=\footnotesize,
   numbersep=9pt,
   tabsize=2,
   breaklines=true,
   showtabs=false,
   captionpos=b
}

\title{How to use web assembly in the browser}
\author{Vinícius Braga Freire}
\date{\today}

\begin{document}
\maketitle

\begin{abstract}
    A small document to share my own experience integrating Web Assembly into the browser.
\end{abstract}


\section{Introduction}

This document aims to present the steps I did to achieve the integration between Web Assembly and the browser in a web application.

Let's first talk a little bit about my environment and what I aimed to do with
Web Assembly. The initial goal was was to run the SMT solver \href{https://cvc5.github.io/}{cvc5} inside a \href{https://ufmg-smite.github.io/proof-visualizer/}{ReactJS application}. Even though the objective was to run it with ReactJS, this tutorial works for plain JS too.

Since cvc5 is written in C++, it was necessary to use a compiling toolchain able
to convert C/C++ code to an assembly code amenable to run in the browser (Web
Assembly). The chosen one was \href{https://emscripten.org/}{Emscripten} since
it has an active community and is used in several projects.

\section{Compiling and Converting}

The first step is to compile the original code into Web Assembly. To do that,
Emscripten offers plenty of tools to compile already existing projects using
their own build systems as a base. Most of these tools will work as wrappers to
your original build system commands (configuration or compilation). For example:
\begin{itemize}
    \item \textbf{emconfigure} configure.sh ...
    \item \textbf{emcmake} cmake ...
    \item \textbf{emmake} make ...
\end{itemize}

If you don't use a build system and compiles manually you can use instead Emscripten compilers \textbf{emcc} and \textbf{em++}.

Since the goal of this document is not to detail the compilation workarounds and
problems I had, feel free to look at
\href{https://github.com/cvc5/cvc5/pull/9006}{my pull request} in the cvc5
repository to see how I managed to modify the cvc5 build system to support Web
Assembly compilation.

\section{Integrating}

When compiling cvc5 to Web Assembly we can obtain 3 versions of the program: WASM (.wasm file for cvc5), JS (not only the wasm, but the .js glue code for web integration) and HTML (both the last two files and also an .html file which supports running the glue code). In this section we will focus on the JS version, since we want to understand how to modify the glue code to satisfy our needs.

Before starting, it's important to keep in mind that the
\href{https://github.com/cvc5/cvc5/blob/main/configure.sh}{--wasm-flags} (that
is a way to pass the
\href{https://github.com/emscripten-core/emscripten/blob/main/src/settings.js}{Emscripten
  options} to the cvc5 linker) influences how the JS glue code is built. The
example below is based on a .js created with the following flags:
\begin{itemize}
    \item \textbf{-s EXPORTED\_FUNCTIONS=\_main}: Adds \textsf{main()} as an
    exported function, but it's not needed since the \textsf{main()} is already
    called when the Module is called.
    %% HB: not clear what ``when the module is called'' means or why this makes
    %% in to necessary to export main. You need to explain what a module is and
    %% how it relates with what you've been describing.
    \item \textbf{-s EXPORTED\_RUNTIME\_METHODS=ccall,cwrap}: Exports JS functions that allow the user to call a single exported function.
    %% HB: clarify which function is exported, not clear from value of the option
    \item \textbf{-s INCOMING\_MODULE\_JS\_API=arguments}: Adds the argument passing to the Module.
    %% HB: change notation to clarify ``arguments'' is to be replaced by
    %% whatever arguments are passed (if I understand correctly)
    \item \textbf{-s INVOKE\_RUN=1}: Whether we will run the \textsf{main()} function.
    \item \textbf{-s EXIT\_RUNTIME=0}: If 0, the runtime does not quit when
    \textsf{main()} completes (allowing code to run afterwards, for example from
    the browser main event loop).
    %% HB: unclear what this could be used for
    \item \textbf{-s ENVIRONMENT=web}: Signalize that the wasm will run in a web environment.
    \item \textbf{-s MODULARIZE}: Make the glue code as an exportable function, just like a Module.
\end{itemize}

Feel free to take a look at my own modified
\href{https://github.com/ufmg-smite/proof-visualizer/blob/main/frontend/src/wasm/cvc5.js}{glue
  code}. This code has a shortened documentation for each of the changes I've
made in the original file. Search for each comment in the file and you will
reach every change I've made.
%% It'd be good to give and overview of the changes before you start detailing
%% them below.

The \href{https://github.com/ufmg-smite/proof-visualizer/blob/main/frontend/src/wasm/cvc5.js#L43}{first change} needed refers to how the glue code will find your .wasm file. The code bellow shows it:

\begin{lstlisting}[language=JavaScript]
// The base directory to the wasm file
var _scriptDir = window.location.origin + '/proof-visualizer/';
\end{lstlisting}

Since this line is outside the Module, it's called only once when the web application that is importing this glue code file is booted. This make sure we will search the file in the correct directory. Since this example is used with ReactJS, then the binary file should be found in the \textbf{public folder} where all the binaries and chunks are stored when the application is built. If you're dealing with plain javascript, the wasm file can go anywhere.

You can observe that in the \href{https://github.com/ufmg-smite/proof-visualizer/blob/main/frontend/src/wasm/cvc5.js#L45-L46}{definition of our Module} (where everything will happen) we receive a Module as argument just like bellow:

\begin{lstlisting}[language=JavaScript]
// This function is responsible for running the web assembly
return function (Module) {
\end{lstlisting}

This module received as argument is an object that allows the interface between external code and the glue code execution, where each element inside it will represent something inside the glue code. For example, we can pass an output and error buffer (that exists in an external scope of this code) and then use these buffers to store the result of the \textsf{std::cout} and \textsf{std:cerr} from the C++ environment. The \href{https://github.com/ufmg-smite/proof-visualizer/blob/main/frontend/src/wasm/cvc5.js#L116-L121}{code bellow} shows how we stored these external buffers inside the glue code:

\begin{lstlisting}[language=JavaScript]
// The out variable is the function that updates the stdout variable
// (it's in the react scope) and its argument is a string (the content
//  of the current stdout line to be written)
var out = Module['out'];
// The err is the same thing as out, but related to stderr
var err = Module['err'];
\end{lstlisting}

If you \href{https://github.com/ufmg-smite/proof-visualizer/blob/main/frontend/src/components/VisualizerSmtDrawer/VisualizerSmtDrawer.tsx#L135-L136}{look outside this glue code} (in the scope where we call the Module function), we have the definition of the 'out' and 'err' just like bellow:

\begin{lstlisting}[language=JavaScript]
const updateStdout = (str: string) => (stdoutRef.current += str + '\n');
const updateStderr = (str: string) => (stderrRef.current += str + '\n');
\end{lstlisting}

With that defined, we simply pass those buffer functions to the Module function \href{https://github.com/ufmg-smite/proof-visualizer/blob/main/frontend/src/components/VisualizerSmtDrawer/VisualizerSmtDrawer.tsx#L357-L366}{just like bellow}:
\begin{lstlisting}[language=JavaScript]
// Run cvc5
Module({
    arguments: args,
    proofTxt: textRef.current,
    out: updateStdout,
    err: updateStderr,
    postcvc5: postcvc5run,
    cleanStdout: cleanStdout,
    binaryFileName: 'cvc5.wasm',
});
\end{lstlisting}

\href{https://github.com/ufmg-smite/proof-visualizer/blob/main/frontend/src/wasm/cvc5.js#L124-L127}{The next slice of code} saves the arguments passed from outside. Those arguments are the arguments of the C++ function you're running:

\begin{lstlisting}[language=JavaScript]
// Assign the arguments passed through the Module (defined in the react
// scope) to a local variable. These arguments are arranged in an array
// of strings
if (Module['arguments']) arguments_ = Module['arguments'];
\end{lstlisting}

Since the function we are running is the cvc5 main function, the arguments passed are basically the argv. \href{https://github.com/ufmg-smite/proof-visualizer/blob/main/frontend/src/components/VisualizerSmtDrawer/VisualizerSmtDrawer.tsx#L41-L42}{Here} we have and example of default arguments:

\begin{lstlisting}[language=JavaScript]
// The default arguments used in the proof generation
const defaultArgs = ['--dump-proofs', '--proof-format=dot', '--proof-granularity=theory-rewrite', '--dag-thresh=0'];
\end{lstlisting}

If you observe above in the call of Module we pass an element called \textbf{binaryFileName} with the value ’cvc5.wasm’. This variable is used inside the glue code to identify, inside the directory we defined in the beginning of this section, the .wasm file we want to run. It's shown next how this  \href{https://github.com/ufmg-smite/proof-visualizer/blob/main/frontend/src/wasm/cvc5.js#L312-L313}{variable of the Module} is used:

\begin{lstlisting}[language=JavaScript]
// Definition of the binary file name
wasmBinaryFile = Module['binaryFileName'];
\end{lstlisting}

After these set ups of variables, an important modification needed is making sure the program can read std::cin the way we need. \href{
https://github.com/ufmg-smite/proof-visualizer/blob/main/frontend/src/wasm/cvc5.js#L730-L743}{Here} we just need to pass std::cin text as an element of the input Module. That way we can simply modify the \href{https://github.com/ufmg-smite/proof-visualizer/blob/main/frontend/src/wasm/cvc5.js#L726}{input read function} and substitute the older way of reading std::cin (most of the time it will be an string returned from an prompt just like var result = window.prompt()). The code bellow show the part of the function that was modified:

\begin{lstlisting}[language=JavaScript]
// Read the smt text send through the Module and
//  pass it to the web assembly binary program as a
//  'stdin' buffer
result = Module['proofTxt'];
// Call a function that cleans the 'stdout' variable
//  in the react scope only a single time (at the
//  first time it's called).
// It's necessary becasuse cvc5, when running in the
//  browser, identify at first time that there is no
//  content in the 'stdin', then it generates an
//  default cvc5 message about the library. This
//  function only make sure that this 'trash' message
//  is removed from the output.
Module['cleanStdout']();
\end{lstlisting}

In the end, when the web assembly finally finishes it calls a function called postRun, that deals with the post run methods. We can utilize this function to do any behavior post run we want. In this case in specific since the run of the wasm is async, \href{https://github.com/ufmg-smite/proof-visualizer/blob/main/frontend/src/wasm/cvc5.js#L264-L270}{we used this function} to call an external function postcvc5 that signalize to the ReactJS that the run is over so it can use the result in the out buffer as it please. The code bellow shows us it:

\begin{lstlisting}[language=JavaScript]
// Function called after the cvc5 execution and no error happend
function postRun() {
    callRuntimeCallbacks(__ATPOSTRUN__);
    // Handle with the post cvc5 (end of execution) operations in the
    // react scope. Signs that no error happend.
    Module['postcvc5'](false);
}
\end{lstlisting}

In a case where an error happens during the execution of the assembly file, then the \href{https://github.com/ufmg-smite/proof-visualizer/blob/main/frontend/src/wasm/cvc5.js#L420-L429 }{\textbf{handleException} function} is called. You can observe bellow that it also calls the function \textbf{Module['postcvc5']}, but now it pass an argument signalizing that an error happened.

\begin{lstlisting}[language=JavaScript]
// Function called after the cvc5 execution when an error happend
function handleException(e) {
    if (e instanceof ExitStatus || e == 'unwind') {
        return EXITSTATUS;
    }
    // Handle with the post cvc5 (end of execution) operations in the
    // react scope and signs that a error happend.
    Module['postcvc5'](true);
    quit_(1, e);
}
\end{lstlisting}

With everything ready, now we only need to care about the start of the run. \href{https://github.com/ufmg-smite/proof-visualizer/blob/main/frontend/src/wasm/cvc5.js#L3674-L3702}{Here} you can see the function responsible for starting the whole process:

\begin{lstlisting}[language=JavaScript]
// Function responsable for running the cvc5 web assembly code with the
//  arguments passed to the Module.
function run(args) {
    // Get the arguments passed to the Module (an array of strings)
    args = args || arguments_;
    if (runDependencies > 0) {
        return;
    }
    preRun();
    if (runDependencies > 0) {
        return;
    }
    function doRun() {
        if (calledRun) return;
        calledRun = true;
        Module['calledRun'] = true;
        if (ABORT) return;
        initRuntime();
        preMain();
        readyPromiseResolve(Module);
        // Call the execution of the main function
        if (shouldRunNow) callMain(args);
        // Call the post run function when no error happen
        postRun();
    }
    {
        doRun();
    }
}
\end{lstlisting}

No changes were made to this function, but it is important to check it to understand how the arguments are passed to the binary and how the \textsf{postRun} is called.

With all of that we are now able to run cvc5 in the browser. Note however that
this process can be applied to any other C/C++ project.

\end{document}
